\documentclass[table]{beamer}
%%\documentclass[handout]{beamer}

\mode<presentation>
{
  \definecolor{navitialight}{RGB}{126,186,200}
  \definecolor{navitiadark}{RGB}{76,102,114}

  \useoutertheme[subsection=false]{miniframes}
  %%\useoutertheme[footline=authortitle]{miniframes}
  \usecolortheme[named=navitiadark]{structure}
  %%\usecolortheme{dolphin}
  \usecolortheme{orchid}
  \useinnertheme{circles}
  \setbeamerfont{block title}{size=\normalsize}
  \setbeamercovered{transparent}

  %%% le foot pour avoir la numérotation des slides %%%
  \setbeamertemplate{footline}{%
    \leavevmode%
    \hbox{%
      \begin{beamercolorbox}[wd=.5\paperwidth,ht=2.5ex,dp=1.125ex,
        leftskip=.3cm plus1fill,rightskip=.3cm]{author in head/foot}%
        \usebeamerfont{title in head/foot}\insertshorttitle
      \end{beamercolorbox}%
      \begin{beamercolorbox}[wd=.5\paperwidth,ht=2.5ex,dp=1.125ex,
        leftskip=.3cm,rightskip=.3cm plus1fil]{title in head/foot}%
        \usebeamerfont{author in head/foot}\insertshortauthor\hfill
        \insertframenumber/\inserttotalframenumber
      \end{beamercolorbox}%
    }%
    \vskip0pt%
  }

  \setbeamercolor{palette primary}{fg=white,bg=navitiadark}
  \setbeamercolor{palette secondary}{fg=white,bg=navitialight}
  \setbeamercolor{palette tertiary}{fg=white,bg=navitiadark}
  \setbeamercolor{palette quaternary}{fg=white,bg=navitialight}
}

\mode<handout>
{
  \usepackage{pgfpages}
  \pgfpagesuselayout{4 on 1}[a4paper,border shrink=5mm,landscape]
}

\usepackage[utf8]{inputenc}
\usepackage{lmodern}
\usepackage[T1]{fontenc}
\usepackage[english,francais]{babel}
\usepackage{multirow}
\usepackage{hhline}

\newcommand{\nologo}{\setbeamertemplate{logo}{}}

\newenvironment{foreignpar}[1][english]{%
    \em\selectlanguage{#1}%
}{}
\newcommand*{\foreign}[2][english]{%
    \emph{\foreignlanguage{#1}{#2}}%
}

\title[BANO + OSM + Navitia]{BANO + OSM + Navitia = un nouveau géocodeur pour les transports}

\author{Guillaume Pinot \and Noémie Lehuby \and Pascal Rhod}

\institute[Kisio Digital] % (optional, but mostly needed)
{
  Kisio Digital\\
  20 rue Hector Malot\\
  75012 Paris, France}
%% - Use the \inst command only if there are several affiliations.
%% - Keep it simple, no one is interested in your street address.

\date{State of the Map France 2017}
%% - Either use conference name or its abbreviation.
%% - Not really informative to the audience, more for people (including
%%   yourself) who are reading the slides online

%% If you have a file called "university-logo-filename.xxx", where xxx
%% is a graphic format that can be processed by latex or pdflatex,
%% resp., then you can add a logo as follows:
\pgfdeclareimage[width=.2\linewidth]{logo}{images/logo_nio}
\logo{\pgfuseimage{logo}\hspace{.04\linewidth}}


%% Delete this, if you do not want the table of contents to pop up at
%% the beginning of each subsection:
\AtBeginSection[]
{
  \begin{frame}<beamer>
    \frametitle{Table des matières}
    \tableofcontents[currentsection,hideothersubsections]
  \end{frame}
}
\AtBeginSubsection[]
{
  \begin{frame}<beamer>
    \frametitle{Table des matières}
    \tableofcontents[currentsection,subsectionstyle=show/shaded/hide]
  \end{frame}
}


\begin{document}

\begin{frame}
  \titlepage
\end{frame}

\section{Pourquoi?}

\begin{frame}
  \frametitle{navitia}

  \begin{itemize}
  \item le cœur de l'information voyageur de Kisio Digital
  \item une API REST
    \begin{itemize}
    \item calcul d'itinéraire multimodal
    \item fiche horaires
    \item prochains passages
    \item autocompletion et géocodage inverse
    \end{itemize}
  \item disponible via \url{https://navitia.io}, notre service ouvert
  \end{itemize}
\end{frame}

\begin{frame}
  \frametitle{Nouveaux besoins autour du géocodage}

  \begin{itemize}
  \item autocompletion sans \foreign{coverage} potentiellement sur le monde
  \item prise en compte de la localisation de l'utilisateur
  \item gestion des zones d'arrêt en fonction des droits de l'utilisateur
  \item séparation en (plus de) microservice
  \end{itemize}
\end{frame}

\begin{frame}
  \frametitle{Et l'existant?}

  \begin{itemize}
  \item NAViTiA 1 (custom, préfix, phonétique, distance)
  \item navitia 2 (custom, préfix, 3gram)
  \item photon (elasticsearch, distance, geocodejson)
  \item addok (redis, phonétique, distance, geocodejson)
  \item geocoder-tester
  \end{itemize}
\end{frame}

\section{Les données}

\begin{frame}
  \frametitle{OpenStreetMap: les villes}

  Utilisation des villes, arrondissement, pays... (\foreign{administrative regions}):
  \begin{itemize}
  \item \foreign{level} configurable
  \item utilisation INSEE, code postaux
  \item import des contours géographiques
  \item coordonées grâce au \foreign{role} \texttt{center} ou au
    barycentre du contour
  \end{itemize}
\end{frame}

\begin{frame}
  \frametitle{OpenStreetMap: les rues}

  Utilisation des rues:
  \begin{itemize}
  \item regroupement des \foreign{ways} grâce à
    \texttt{associatedStreet} ou par ville
  \item affectation de la ville grâce aux contours géographiques
  \item coordonées grâce à un point des \foreign{ways}
  \end{itemize}
\end{frame}

\begin{frame}[fragile]
  \frametitle{OpenStreetMap: les POI}

  \begin{itemize}
  \item configuration à la lecture d'OSM
  \item type de POI
  \item si pas de \texttt{name}, utilisation du type de POI
  \item coordonées en fonction du type de l'objet
  \end{itemize}

  \tiny
\begin{verbatim}
{
  "poi_types": [
    {"id": "amenity:school", "name": "École"},
    {"id": "amenity:bar_tobacco", "name": "Bar-tabac"},
  ],
  "rules": [
    {
      "osm_tags_filters": [{"key": "amenity", "value": "college"}],
      "poi_type_id": "amenity:school"
    },
    {
      "osm_tags_filters": [{"key": "amenity", "value": "university"}],
      "poi_type_id": "amenity:school"
    },
    {
      "osm_tags_filters": [{"key": "amenity", "value": "cafe"}, {"key": "tobacco", "value": "yes"}],
      "poi_type_id": "amenity:bar_tobacco"
    }
  ]
}
\end{verbatim}
\end{frame}

\begin{frame}
  \frametitle{BANO: les adresses}

  \begin{itemize}
  \item import direct des adresses
  \item rattachement des régions administratives par INSEE et contours
  \end{itemize}
\end{frame}

\begin{frame}
  \frametitle{GTFS: les zones d'arrêt}

  \begin{itemize}
  \item import des zones d'arrêt uniquement (\texttt{location\_type=1})
  \item importance par nombre de point d'arrêt
  \item rattachement des régions administratives par contours
  \end{itemize}
\end{frame}

\section{La recherche}

\begin{frame}
  \frametitle{La recherche}

  \begin{itemize}
  \item geocodejson
  \item mot + prefix + 3gram + type d'objet + code postal + importance
  \item priorisation optionelle par coordonées
  \item filtre optionnel par contour géographique
  \end{itemize}
\end{frame}

\section{Retour d'expérience}

\begin{frame}
  \frametitle{Adresses et rues dans OSM}

  Les adresses et les rues dans OSM, c'est compliqué:
  \begin{itemize}
  \item pleins de modérilations
  \item décorélation explicite de la ville
  \item code postaux
  \end{itemize}
\end{frame}

\begin{frame}
  \frametitle{Problèmes de contributions}

  Les données contiennent des problèmes de contributions:
  \begin{itemize}
  \item \foreign{boundaries} cassées
  \item rues mal regroupées
  \end{itemize}
\end{frame}

\begin{frame}
  \frametitle{BANO}

  \begin{itemize}
  \item plus d'évolution
  \item pas de stratégie de fusion BAN/BANO
  \end{itemize}
\end{frame}

\begin{frame}
  \frametitle{Technique}

  \begin{itemize}
  \item Rust
  \item elasticsearch
  \end{itemize}
\end{frame}

\section{Conclusion}

\begin{frame}
  \frametitle{Conclusion}

\end{frame}

\begin{frame}
  \titlepage
\end{frame}

\end{document}
