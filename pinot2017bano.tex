\documentclass[table]{beamer}
%%\documentclass[handout]{beamer}

\mode<presentation>
{
  \definecolor{navitialight}{RGB}{126,186,200}
  \definecolor{navitiadark}{RGB}{76,102,114}

  \useoutertheme[subsection=false]{miniframes}
  %%\useoutertheme[footline=authortitle]{miniframes}
  \usecolortheme[named=navitiadark]{structure}
  %%\usecolortheme{dolphin}
  \usecolortheme{orchid}
  \useinnertheme{circles}
  \setbeamerfont{block title}{size=\normalsize}
  \setbeamercovered{transparent}

  %%% le foot pour avoir la numérotation des slides %%%
  \setbeamertemplate{footline}{%
    \leavevmode%
    \hbox{%
      \begin{beamercolorbox}[wd=.5\paperwidth,ht=2.5ex,dp=1.125ex,
        leftskip=.3cm plus1fill,rightskip=.3cm]{author in head/foot}%
        \usebeamerfont{title in head/foot}\insertshorttitle
      \end{beamercolorbox}%
      \begin{beamercolorbox}[wd=.5\paperwidth,ht=2.5ex,dp=1.125ex,
        leftskip=.3cm,rightskip=.3cm plus1fil]{title in head/foot}%
        \usebeamerfont{author in head/foot}\insertshortauthor\hfill
        \insertframenumber/\inserttotalframenumber
      \end{beamercolorbox}%
    }%
    \vskip0pt%
  }

  \setbeamercolor{palette primary}{fg=white,bg=navitiadark}
  \setbeamercolor{palette secondary}{fg=white,bg=navitialight}
  \setbeamercolor{palette tertiary}{fg=white,bg=navitiadark}
  \setbeamercolor{palette quaternary}{fg=white,bg=navitialight}
}

\mode<handout>
{
  \usepackage{pgfpages}
  \pgfpagesuselayout{4 on 1}[a4paper,border shrink=5mm,landscape]
}

\usepackage[utf8]{inputenc}
\usepackage{lmodern}
\usepackage[T1]{fontenc}
\usepackage[english,francais]{babel}
\usepackage{multirow}
\usepackage{hhline}

\newcommand{\nologo}{\setbeamertemplate{logo}{}}

\newenvironment{foreignpar}[1][english]{%
    \em\selectlanguage{#1}%
}{}
\newcommand*{\foreign}[2][english]{%
    \emph{\foreignlanguage{#1}{#2}}%
}

\title[BANO + OSM + Navitia]{BANO + OSM + Navitia = un nouveau géocodeur pour les transports}

\author{Guillaume Pinot \and Noémie Lehuby \and Pascal Rhod}

\institute[Kisio Digital] % (optional, but mostly needed)
{
  Kisio Digital\\
  20 rue Hector Malot\\
  75012 Paris, France}
%% - Use the \inst command only if there are several affiliations.
%% - Keep it simple, no one is interested in your street address.

\date{State of the Map France 2017}
%% - Either use conference name or its abbreviation.
%% - Not really informative to the audience, more for people (including
%%   yourself) who are reading the slides online

%% If you have a file called "university-logo-filename.xxx", where xxx
%% is a graphic format that can be processed by latex or pdflatex,
%% resp., then you can add a logo as follows:
\pgfdeclareimage[width=.2\linewidth]{logo}{images/logo_nio}
\logo{\pgfuseimage{logo}\hspace{.04\linewidth}}


%% Delete this, if you do not want the table of contents to pop up at
%% the beginning of each subsection:
\AtBeginSection[]
{
  \begin{frame}<beamer>
    \frametitle{Table des matières}
    \tableofcontents[currentsection,hideothersubsections]
  \end{frame}
}
\AtBeginSubsection[]
{
  \begin{frame}<beamer>
    \frametitle{Table des matières}
    \tableofcontents[currentsection,subsectionstyle=show/shaded/hide]
  \end{frame}
}


\begin{document}

\begin{frame}
  \titlepage
\end{frame}

\section{Introduction}

\begin{frame}
  \frametitle{Introduction}

  Mimir!
\end{frame}

\section{Pourquoi?}

\begin{frame}
  \frametitle{Pourquoi?}

  \begin{itemize}
  \item présentation navitia
  \item besoin autocompletion mondiale
  \item besoins spécifiques (zones d'arrêt)
  \item photon et addok
  \end{itemize}
\end{frame}

\section{Les données}

\begin{frame}
  \frametitle{Les données}

  \begin{description}
  \item[BANO] adresses
  \item[OSM] rues, villes, POI
  \item[stops.txt] zones d'arrêt
  \end{description}
\end{frame}

\section{La recherche}

\begin{frame}
  \frametitle{La recherche}

  \begin{itemize}
  \item geocodejson
  \item prefix + 3gram + type d'objet + code postal
  \item from
  \item shape
  \end{itemize}
\end{frame}

\section{Retour d'expérience}

\begin{frame}
  \frametitle{Adresses et rues dans OSM}

  Les adresses et les rues dans OSM, c'est compliqué:
  \begin{itemize}
  \item pleins de modérilations
  \item décorélation explicite de la ville
  \item code postaux
  \end{itemize}
\end{frame}

\begin{frame}
  \frametitle{Problèmes de contributions}

  Les données contiennent des problèmes de contributions:
  \begin{itemize}
  \item \foreign{boundaries} cassées
  \item rues mal regroupées
  \end{itemize}
\end{frame}

\begin{frame}
  \frametitle{BANO}

  \begin{itemize}
  \item plus d'évolution
  \item pas de stratégie de fusion BAN/BANO
  \end{itemize}
\end{frame}

\begin{frame}
  \frametitle{Technique}

  \begin{itemize}
  \item Rust
  \item elasticsearch
  \end{itemize}
\end{frame}

\section{Conclusion}

\begin{frame}
  \frametitle{Conclusion}

\end{frame}

\begin{frame}
  \titlepage
\end{frame}

\end{document}
